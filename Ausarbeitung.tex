\documentclass{scrreprt}
\usepackage{datagidx}
\usepackage{lstdoc} % Ausarbeitung
% SW
\setcounter{tocdepth}{3}                   % Zeigt Chapter, Section, Subsection in toc
\usepackage{tabularx}
% END SW

\KOMAoptions{DIV      = 10,                % Gibt die Größe des Textbereichs an
             fontsize = 11pt,              % Schriftgröße
             paper    = a4,                % DIN A4-Papier
             parskip  = half}              % Kurzer Abstand statt eingerückter Absatzstart
\setlength{\headheight}{24pt}
\displaywidowpenalty=1000%                 % Leicht erhöhte Strafe für Witwen durch abgesetzten
                                           % Mathematiksatz
\widowpenalty=1000%                        % Leicht erhöhte Strafe für Witwen

\usepackage{iftex}                         % LuaLaTeX wird vorausgesetzt (für fontspec und selnolig)
\RequireLuaTeX%

\usepackage{ragged2e}                      % Für Blocksatz
\usepackage{booktabs}                      % Bessere Tabellenlinien (eigene Befehle, Doku hilft)
\usepackage{csquotes}                      % Bessere Zitate (eigene Befehle, Doku hilft)
\usepackage{relsize}                       % Abgesetzte Zitate mit verkleinerter Schrift
\newenvironment{smallquotes}{\quote\smaller}{\endquote}
\SetBlockEnvironment{smallquotes}
\usepackage{ellipsis}                      % Korrigiert Platz nach \dots
\usepackage{etoolbox}                      % Um das ellipsis-Paket robust zu machen
\robustify\textellipsis%
\usepackage{enumerate}                     % Weitere Möglichkeiten für die enumerate-Umgebung
\usepackage{geometry}                      % Zentriert die Titelseite für die Druckausgabe
\usepackage{graphicx}                      % Einbinden von Bildern
\graphicspath{Bilder/}                      % Bilder können aus dem Verzeichnis img ohne Pfadangabe
                                           % eingebunden werden
\usepackage{icomma}                        % Korrigiert Platz um Kommas in Zahlen
\usepackage{listings}                      % Einbinden von Quelltexten
\usepackage{xcolor}                        % Nutzung von Farben
\lstset{%
    breaklines       = true,               % Automatischer Zeilenumbruch
    backgroundcolor  = \color{gray!8},
    numbers          = left,
    numbersep        = 5pt,
    numberstyle      = \textcolor{black!40},
    showspaces       = false,
    showstringspaces = false,
    tabsize          = 4,
    basicstyle=\small\ttfamily,
    keywordstyle=\bfseries\ttfamily\color{blue},
    stringstyle=\color{orange!50!black}\ttfamily,
    commentstyle=\color{gray}\ttfamily,
    literate=%
        {Ö}{{\"O}}1
        {Ä}{{\"A}}1
        {Ü}{{\"U}}1
        {ß}{{\ss}}1
        {ü}{{\"u}}1
        {ä}{{\"a}}1
        {ö}{{\"o}}1
        {~}{{\textasciitilde}}1
}

\renewcommand*{\lstlistlistingname}{Listings-Verzeichnis}
\usepackage{mathtools}                     % Große Liste von Mathematikbefehlen, enthält amsmath
\usepackage{fontspec}                      % Für den normalen Schriftsatz (standardmäßig: lmodern)
\usepackage[final]{microtype}              % Verbessert die Lesbarkeit und verringert die Anzahl der
                                           % Trennungen
\usepackage{polyglossia}                   % Trennregeln für Deutsch und Englisch
\setmainlanguage[babelshorthands,          % Kürzel für Trennregeln wie im Paket babel
                 spelling = new]{german}
\setotherlanguage{english}
\usepackage[backend = biber,               % Standardmäßig ab TeXLive 2016
            style   = ieee]{biblatex}      % Bibliographie im IEEE-Stil
\addbibresource{Quellen/Literatur.bib}
% Definiert den Bibliographiestil für Online-Quellen nach IEEE StyleGuide von 2009.
% Unter Umständen muss der Stil aktualisiert werden oder kann entfernt werden, wenn er in den
% BibLaTeX-Stil integriert wird.
% https://www.ieee.org/documents/ieeecitationref.pdf
\DefineBibliographyStrings{german}{%
    url = {\mkbibbrackets{Online}\adddot\addspace{}Verfügbar}
}
\renewbibmacro*{title}{%
    \ifboolexpr{
        test {\iffieldundef{title}\fi}
    }\fi
    {}
    {%
        \printtext[title]{%
            \printfield[sentencecase]{title}%
        }%

    }
    \printfield{titleaddon}%
}
\DeclareBibliographyDriver{online}{%
    \usebibmacro{bibindex}%
    \usebibmacro{begentry}%
    \usebibmacro{author/editor+others/translator+others}%
    \adddot\addspace%
    \mkbibparens{\usebibmacro{date}}%
    \adddot\addspace%
    \usebibmacro{title}%
    \addspace%
    \usebibmacro{url}%
}
\usepackage{scrhack}                       % Verbessert die Kompatibilität zwischen KOMAskript und
                                           % einigen älteren Paketen
\usepackage[automark]{scrlayer-scrpage}    % Seitenlayout anpassen
\cohead*{}
\cofoot*{}
\rofoot*{\pagemark}
\usepackage[ngerman]{selnolig}             % Unterdrückt selektiv Ligaturen
\usepackage[ngerman]{varioref}             % Referenziert Labels wo möglich und nötig mit
                                           % relativen Seitenangaben
\usepackage[breaklinks,                    % Trennung von Links erlauben
            colorlinks,                    % Links farbig setzen
            hyperfootnotes = false,        % Sehr fragil und unnötig
            linkcolor      = black,
            urlcolor       = blue]
           {hyperref}                      % Klickbare Hyperlinks
\usepackage[acronym,                       % Abkürzungsverzeichnis erstellen
            nogroupskip,                   % Kein Abstand zwischen Gruppen
            nonumberlist,                  % Keine nummerierte Liste der Einträge
            nopostdot,                     % Kein automatischer Punkt nach Einträgen
            xindy]
            {glossaries}                    % Zum Einbinden eines Abkürzungsverzeichnis und eines
                                           % Glossars
\GlsSetXdyCodePage{duden-utf8}             % Einstellungen für Xindy: Sortierung und UTF8
\setacronymstyle{short-long}               % Erste Abkürzung im Format kurz (lang)
\setglossarystyle{long}                    % Gleiche Breite für Spalten
\setlength\LTleft\parindent%               % Linke Spalte linksbündig statt zentriert
\setlength\LTright\fill%                   % Rechte Spalte füllt die restliche Seite
\loadglsentries{Quellen/glossary.tex}              % Glossareinträge aus dieser Datei lesen
\makeglossaries%                           % Glossareinträge erstellen
\usepackage{cleveref}                      % Referenziert zusätzlich den Typ eines Labels

\AtBeginDocument{% Folgende Befehle am Beginn des Dokuments ausführen
    \hyphenation{Hash-al-go-rith-mus} % Hier globale Regeln zur Silbentrennung einfügen
    \renewcommand*{\acronymname}{Abkürzungsverzeichnis}
}

% Wenn URLs in der Bibliographie unbedingt an jeder Stelle umgebrochen werden können dürfen müssen,
% können die nachfolgenden Zeilen genutzt werden. Achtung: Sehr hässlich. Das \raggedright von unten
% kann in diesem Fall entfernt werden.
% \setcounter{biburlnumpenalty}{1}
% \setcounter{biburlucpenalty}{1}
% \setcounter{biburllcpenalty}{1}

\AtEndDocument{% Folgende Befehle am Ende des Dokuments ausführen
    \cleardoublepage%
    \raggedright%
    % \sloppypar% Kann als Alternative zum \raggedright eingesetzt werden. Blocksatz wie in Word.
    \printbibliography% Bibliographie ausgeben
}

\newcommand*{\Signature}{\Author\\[-0.3cm]\rule{6cm}{1pt}\\Autor}
\newcommand*{\SignatureZwei}{\AuthorZwei\\[-0.3cm]\rule{6cm}{1pt}\\Autor}

\newcommand*{\command}[1]{\texttt{#1}}  % Befehle innerhalb der Zeile in Schreibmaschinenschrift
                                        % setzen
\newcommand*{\definition}[1]{\emph{#1}} % Definitionen in kursiver Schrift setzen
\newcommand*{\filename}[1]{\texttt{#1}} % Dateinamen in Schreibmaschinenschrift setzen
\newcommand*{\programname}[1]{\gls{#1}} % Programme automatisch ins Glossar aufnehmen (nicht
                                        % unbedingt benötigt)
                                        
%\end{document}
 % Ausgelagerte Präambel

% Hier können die Informationen über die Ausarbeitung eingetragen werden
\newcommand*{\Course}{Projektarbeit} % Bei vorlesungsbegleitenden Ausarbeitungen
\newcommand*{\Subject}{Deepfakes und Social Engineering} % Thema der Ausarbeitung
\newcommand*{\Author}{Julian Faigle}
\newcommand*{\Matriculation}{86292}
\newcommand*{\Discipline}{ITS}
\newcommand*{\AuthorZwei}{Max Ernstschneider}
\newcommand*{\MatriculationZwei}{86464}
\newcommand*{\DisciplineZwei}{AIT}
\newcommand*{\Semester}{6}
\newcommand*{\Professor}{Prof. Roland Hellman}
\newcommand*{\Deadline}{15.08.2024}
\newcommand*{\Location}{Aalen}

% Um die Übersetzung zu beschleunigen, kann man hiermit nur Teile der Ausarbeitung übersetzen lassen
% \includeonly{FILENAME,FILENAME,...}

% Die folgenden Zeilen können für die Druckausgabe verwendet werden
% Achtung: Durch den vergrößerten Textbereich kann eine erneute Optimierung der Zeilenumbrüche
% notwendig sein.
% \KOMAoptions{BCOR     = 0mm,       % Hier die durchschnittliche Länge eintragen, die durch die
%                                    % Bindung verloren geht
%              DIV      = 11,        % Vergrößerter Textbereich für die Druckausgabe
%              headings = openright, % Kapitel starten immer auf der rechten Seite
%              twoside}
% \ofoot*{\pagemark}
% \ifoot*{}
% \renewcommand*{\Signature}{\rule{6cm}{1pt}\\Unterschrift}

% Metadaten für das PDF
\hypersetup{%
    pdfinfo={%
    Author={\Author\ (Matrikelnummer: \Matriculation); \AuthorZwei\ (Matrikelnummer: \MatriculationZwei)},
    Title={\Subject},
    Subject={\Course},
    Keywords={\Course, \Course, \Author, \Discipline, \Professor}
    }
}

\begin{document}
    \newgeometry{hmargin = {0cm, 0cm}} % Sicherstellen, dass die Titelseite zentriert ist
\begin{titlepage}
    \centering% Horizontal zentrieren
    \vspace*{\fill}% Vertikal zentrieren (zusammen mit unten)
    \linespread{1.2}
    \sffamily
    \textbf{\huge{\Course}}\\[1.2cm]
    \large{\textbf{\Subject}}\\[1.2cm]
    vorgelegt von\\[1.2cm]
    \Author\ (Matrikelnummer: \Matriculation)\\
    Studiengang \Discipline\\[1.0cm]
    \AuthorZwei\ (Matrikelnummer: \MatriculationZwei)\\
    Studiengang \DisciplineZwei\\[1.0cm]
    Semester \Semester\\[0.2cm]
    \includegraphics[scale=0.3]{Bilder/htw}\\[0.1cm]
    \textbf{Hochschule Aalen}\\[0.2cm]
    Hochschule für Technik und Wirtschaft\\[0.2cm]
    Betreut durch \Professor\\[0.2cm]
    \Deadline%
    \vspace*{\fill}% Vertikal zentrieren (zusammen mit oben)
\end{titlepage}
\restoregeometry% Bisheriges Seitenformat wiederherstellen
      % Titelseite
    \cleardoublepage%
\pagestyle{plain}
\pagenumbering{Roman}

\chapter*{Erklärung}
Wir versichern, dass wir die Ausarbeitung mit dem Thema \enquote{\Subject} selbstständig verfasst haben und keine anderen Quellen und Hilfsmittel als die angegebenen benutzt haben.
Die Stellen, die anderen Werken dem Wortlaut oder dem Sinn nach entnommen wurden, sind in jedem einzelnen Fall unter Angabe der Quelle als Entlehnung (Zitat) kenntlich gemacht worden.
Das Gleiche gilt für beigefügte Skizzen und Darstellungen.\\[4cm]
\parbox{\textwidth}{%
    \parbox[b]{7cm}{%
        \Location, den \today\\[-0.3cm]
        \rule{6cm}{1pt}\\
        Ort, Datum
    }
    \hfill%
    \parbox[b]{7cm}{%
        \Signature%
    }
}
\\[0.5cm]
\parbox{\textwidth}{%
    \hfill%
    \parbox[b]{7cm}{%
        \SignatureZwei%
    }
}
 % Eidesstattliche Erklärung

    % Hier eigene Kapitel einfügen

    \chapter{Kurzfassung}\label{ch:kurzfassung}

In den letzten Jahren haben technologische Fortschritte im Bereich der \gls{ki} und des maschinellen Lernens zu einer signifikanten Entwicklung in der digitalen Manipulation von Medien geführt.
Insbesondere Deepfakes, bei denen KI verwendet wird, um realistisch aussehende Bilder, Videos oder Audiodateien zu erstellen, die es so nie gegeben hat, haben die Aufmerksamkeit von Forschern, Sicherheitsexperten und der breiten Öffentlichkeit gleichermaßen erregt.
Diese Technologie, die ursprünglich zur Verbesserung von visuellen Effekten und für kreative Zwecke entwickelt wurde, hat sich mittlerweile zu einem potenten Werkzeug entwickelt, das auch für bösartige Zwecke missbraucht werden kann.

Deepfakes basieren auf komplexen Algorithmen wie Generative Adversarial Networks (GANs), die es ermöglichen, Gesichter in Videos nahtlos auszutauschen, Stimmen zu imitieren oder Personen in Szenarien darzustellen, die nie stattgefunden haben.
Diese Fähigkeit, die Realität auf eine Weise zu verzerren, die für das menschliche Auge oft schwer zu erkennen ist, birgt erhebliche Risiken für die Gesellschaft.
Von der Verbreitung von Fehlinformationen über politische Manipulationen bis hin zu Erpressungen und Identitätsdiebstahl – die Einsatzmöglichkeiten von Deepfakes sind vielfältig und gefährlich.

Gleichzeitig hat Social Engineering, die Kunst der Manipulation von Menschen, um vertrauliche Informationen zu erhalten oder bestimmte Aktionen auszulösen, eine neue Dimension erreicht.
Social Engineering nutzt gezielt psychologische Techniken, um das Vertrauen von Individuen zu gewinnen und sie dazu zu bringen, sicherheitsrelevante Fehler zu begehen.
In einer zunehmend digitalisierten Welt, in der persönliche und berufliche Interaktionen häufig online stattfinden, wird die Verbindung zwischen Social Engineering und digitalen Technologien immer enger.

Die Kombination von Deepfakes und Social Engineering stellt eine besonders gefährliche Bedrohung dar.
Angreifer können Deepfakes nutzen, um das Vertrauen ihrer Zielpersonen zu erschleichen oder deren Entscheidungen zu beeinflussen, indem sie manipulierte Inhalte einsetzen, die beispielsweise vertraute Gesichter oder Stimmen imitieren.
Solche Angriffe sind schwer zu erkennen und zu bekämpfen, da sie sowohl technisches Wissen als auch ein tiefes Verständnis menschlicher Verhaltensmuster erfordern.

In dieser Ausarbeitung werden verschiedene Technologien im Bereich Video- und Audio-Deepfakes beleuchtet und vorgestellt.
Sie werden in den Kontext von Social Engineering eingeordnet und praktisch angewendet, um beispielhafte Szenarien darzustellen % Kurzfassung

    \include{tex/frontmatter}    % Verzeichnisse, Vorbereitung Layout eigene Kapitel

    % \input{}
Test \glspl{wlan} huhu \gls{lan} \\
\cite{google} \\ % Kapitel

\end{document}
