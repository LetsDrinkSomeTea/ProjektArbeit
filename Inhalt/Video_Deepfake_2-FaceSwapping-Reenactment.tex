\section{Face Swapping und Reenactment}\label{sec:face-swapping-und-reenactment}
Es gibt verschiedene Techniken von Video Deepfakes, im Folgenden werden Face-Swapping, sowie Reenactment näher betrachtet.
Beide Varianten können mit den oben vorgestellten Möglichkeiten realisiert werden.
Es gibt Modelle die für einen von beiden Anwendungsfällen besser geeignet sind, grundlegend basieren aber heutige Modelle immer auf \glspl{gan}.

\subsection*{Face Swapping}\label{subsec:face-swapping}
Face Swapping, eine weit verbreitete Technik innerhalb der Deepfake-Technologie, beinhaltet das Austauschen eines
Gesichts in einem Bild oder Video durch das Gesicht einer anderen Person.
Diese Methode hat insbesondere in den letzten Jahren erhebliche Fortschritte gemacht, vor allem durch die Entwicklung von \glspl{gan}.
Bei Face Swapping wird das Gesicht der Zielperson durch ein anderes Gesicht ersetzt, wobei Merkmale wie Hautfarbe,
Beleuchtung und Gesichtsausdrücke so angepasst werden, dass das Ergebnis möglichst realistisch wirkt.
Diese Technik findet vor allem Anwendung in der Gestaltungs- und Medienbranche.
Es können z.B. Gesichter von Schauspielern auf ihre Stuntdoubles gesetzt werden, um realistischere Stunt Szenen zu erzeugen.
Eine besondere Form des Face Swapping ersetzt speziell den Mund eines Schauspielers, um die Synchronisation in anderen Sprachen zu vereinfachen\cite{Deepfakes-An-Overview}.
Da nur das Gesicht ersetzt wird, müssen Dinge wie Hintergrund, Frisur und Kleidung selbst an die Zielperson angepasst werden.
Dieser Aufwand ist heutzutage nicht mehr nötig, da Reenactmentmodelle ähnlich gute Ergebnisse erzielen.
Bei hochrangigen Zielen lohnt sich dieser Mehraufwand, da die Ergebnisse von Face Swapping zum heutigen Stand der Technik noch unübertroffen sind.

\subsection*{Reenactment}\label{subsec:reenactment}
Reenactment, auch als Face Transfer oder Puppeteering bekannt, ist eine Technik, bei der die Gesichtsausdrücke und -bewegungen eines Ausgangsbildes oder -videos auf ein Zielbild oder -video übertragen werden.
Dies ermöglicht es, das Gesicht der Zielperson so zu manipulieren, dass es die gleichen Bewegungen und Ausdrücke wie das Ausgangsgesicht zeigt.
Diese Technik findet sich ebenfalls in der Filmbranche wieder, indem z.B. verstorbene oder anderweitig verhinderte Schauspieler trotzdem noch in Filmen oder Serien zu sehen sind.
Das vermutlich bekannteste Beispiel hierfür ist die Nutzung von Deepfake-Technologie, um die Charaktere Grand Moff Tarkin und Prinzessin Leia in ``Rogue One: A Star Wars Story'' realistischer darzustellen.
In der originalen Filmproduktion wurden von beiden Charakteren alte 3D-Modelle bzw. Facescans verwendet um mit herkömmlichen Animations- und Rendertechniken realisiert.
Durch den Einsatz von Face-Swapping wurde das Reenactment dieser Charaktere von Fans so verbessert, dass sie natürlicher und lebensechter wirken als die ursprünglichen Effekte.
Dieses Beispiel zeigt die Vorteile vom Einsatz von Deepfakes in der Filmproduktion und erweitern die Möglichkeiten der Branche erheblich\cite{rouge-one-deepfake}.\\
Im Kontext von Cyber-Security sind die Anwendungsfälle offensichtlich.
Es können durch Reenactment Videos von einflussreichen Personen innerhalb von Firmen erstellt werden, um Phishing noch effektiver zu gestalten.
Außerdem können Videos von Personen des öffentlichen Lebens angefertigt werden in denen diese kontroverse Aussagen tätigen, um die öffentliche Meinung ins Negative zu ziehen.
Ein gutes Beispiel hierfür ist \href{https://www.youtube.com/watch?v=cQ54GDm1eL0}{``You Won’t Believe What Obama Say In This Video''}.\\
Vor allem durch den Einsatz von auf Performance spezialisierter \glspl{gan} können Videos nahezu in Echtzeit gefaked werden.
Für Social Engineering relevante Videomedien sind ohnehin Video-Calls -- dies hat zur Folge, dass ein Delay von wenigen Sekunden, sowie kleine Artefakte oder Bildrauschen nicht ins Gewicht fallen, da diese auch von der Streamingplattform ausgehen könnten.
Die Qualität der Deepfakes braucht also nicht auf filmreifen Niveau zu sein, um für Social Engineering brauchbar zu sein.
Dies hat zur Folge, dass schon mit wenig Aufwand und Wissen, eine Großzahl von Personen solche Fakes erstellen können.