\section{Social Engineering}\label{sec:social_engineering}
Unter Social Engineering werden alle Angriffe zusammengefasst, die die Schwachstelle Mensch ausnutzen.
Es wird durch verschiedene Techniken versucht an private oder sensible Inhalte von Personen zu kommen.
Social Engineering ist heutzutage eine der größten Gefahren im digitalen Raum.
Kryptografische Verfahren und Protokolle wurden über die Jahre immer besser.
Ist ein System bzw\. eine digitale Infrastruktur richtig gehärtet, sind Angriffe wie Brute-Force oder Dictionary
-Attacks wirkungslos.
Außerdem werden durch moderne \glspl{ids}, sowie \gls{threat-intelligence} technische Angriffe immer schneller
erkannt und blockiert.
Laut einem Paper aus 2018, sind 84\% aller Cyber-Angriffe auf Social Engineering zurückzuführen.
Zudem haben Social Engineering Angriffe eine höhere Erfolgschance als herkömmliche Methoden.
Laut des \gls{dbir} von Verizon waren 2024 \[45\%\] der erfassten Cyberangriffe auf Social Engineering zurückzuführen.
Bei \[~83\%\] der 3661 reporteten Incidents Daten extrahiert.\cite{verizon-2024-dbir}

\subsection{Verschiedene Typen von Social Engineering}\label{subsec:verschiedene-typen-von-social-engineering}
Social Engineering lässt sich in verschiedene Bereiche untergliedern, die folgenden Beschreibungen richten sich nach
dem Artikel: ``Social Engineering Attacks: A Survey''\cite{social-engineering-a-survey}

Hier werden Social Engineering Angriffe in zwei Kategorien unterteilt.
\begin{itemize}
    \item Human-based
\end{itemize}
    Diese Angriffe werden manuell von einem Menschen ausgeführt.
    Sie sind i.d.R\. spezifisch auf das Opfer angepasst und mit höherem Aufwand verbunden.
    Dafür sind die Erfolgschancen höher als bei automatisierten Angriffen.
\begin{itemize}
    \item computer-based
\end{itemize}
Diese Angriffe werden automatisiert durchgeführt.
Sie sind qualitativ deutlich schlechter als ihr Counterpart, dafür werden sie in hoher Quantität durchgeführt.
Hierzu zählen Phishing-Mails oder SMS\@.
Es gibt verschiedene Tools um solche Angriffe durchzuführen, ein bekanntes ist das ``Social Engineering Toolkit''\href{https://github.com/trustedsec/social-engineer-toolkit}

Des Weiteren können Social Engineering Angriffe in drei weitere Kategorien unterteilt werden.


Außerdem noch die Unterteilung in drei Kategorien abhängig davon wie der Angriff ausgeführt wird
- social
    Zwischenmenschliche Interaktion, z.B\. baiting und spear phishing
- technical
    Angriff durchs Internet/Online Services um Passwörter, cc details, usw.
- physical
    Angriffe in RL, z.B \. Mülleimer nach Dokumenten durchsuchen

-based attacks


Weitere Klassifizierung:
- direkt
    Direkter Kontakt zwischen Angreifer und Opfer, z.B\. Shoulder Surfing, Dumpster Diving, impersonation on helpdesk calls
- indirekt
    Kein direkten Kontakt, Angriffe können asynchron und remote ausgeführt werden z.B\. Phishing, ransomware, fake software

Kleiner Überblick über manche Techniken:
# Phishing
    Meist verbreiteter Social engineering Angriff, mit dem Ziel private oder vertrauliche Daten der Opfer zu stehlen.
    Dabei wird werden vor allem Mails, SMS, Anrufe oder fake Webseiten eingesetzt.

Phishing lässt sich grob unterteilen in:
- Spear Phishing
    Zielspezifisch, durch z.B\. OSINT angepasste Mails
- Whaling Phishing
    Angriffe auf hochrangige Ziele
- Vishing
    Voice Phishing via z.B\. Telefon, Teams, usw.


# Baiting
Baiting, auch Road Apples genannt sind Phishing Angriffe die jemanden dazu verleiten sollen auf etwas zu klicken um
etwas gratis zu bekommen. Auch Bad-USBs gehören dazu.

# Tailgating Attacks
Ist das unerlaubte Zutritt verschaffen indem z.B\. einer autorisierten Person gefolgt wird, RFID Chips gehackt werden.

Gegenmaßnahmen:
Vor allem Schulung der Mitarbeiter. Aber auch das überprüfen der Authentizität und Integrität von Nachrichten wie
Mails usw.
Außerdem vor allem Schadensbegrenzung, human-based social attacks sind schwer/unmöglich automatisiert zu erkennen.
Durch Techniken wie z.B\. Domain-Tiering können die Worst-Case Szenarien abgeschwächt werden.
Notfallmanagement ist ebenfalls entscheidend.