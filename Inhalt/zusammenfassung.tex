\section*{Social Engineering}

Social Engineering ist eine Methode, die gezielt menschliches Verhalten manipuliert, um unautorisierten Zugang zu Informationen oder Systemen zu erlangen.
Die Ausarbeitung hebt hervor, dass Angriffe wie Phishing, Pretexting und Baiting besonders effektiv sind, da sie auf psychologische Tricks setzen, um Benutzer zur Preisgabe sensibler Informationen zu verleiten.
Phishing beispielsweise wird häufig durch E-Mails oder Nachrichten ausgeführt, die vorgeben, von vertrauenswürdigen Quellen zu stammen.
Pretexting wiederum nutzt falsche Identitäten, um gezielt Informationen zu erlangen.
Diese Angriffe sind besonders gefährlich, da sie unabhängig von technischen Sicherheitsmaßnahmen erfolgreich sein können und auf die Schwachstelle Mensch abzielen.

\section*{Deepfakes}

Deepfakes stellen eine moderne Bedrohung dar, die durch den Einsatz von \gls{ki} und Maschinellem Lernen (ML) ermöglicht wird.
Durch die Verwendung von \glspl{gan} können täuschend echte Videos und Bilder erstellt werden, die kaum von echten Inhalten zu unterscheiden sind.
In der Ausarbeitung werden auf Risiken hingewiesen, die mit Deepfakes einhergehen, da sie zur Verbreitung von Desinformation, zur Rufschädigung und zur Manipulation der öffentlichen Meinung genutzt werden können.
Die Fähigkeit, visuelle Inhalte auf diese Weise zu fälschen, stellt eine erhebliche Herausforderung für die Wahrnehmung und das Vertrauen in digitale Medien dar.

\section*{Erkennungsmechanismen von Deepfakes}

Die Erkennung von Deepfakes ist ein komplexes Unterfangen, da die Technologie stetig fortschreitet und die Ergebnisse immer realistischer werden.\ In der Ausarbeitung werden verschiedene Erkennungsmethoden beschrieben, wie die Analyse von Blinkmustern, die Überprüfung von Unregelmäßigkeiten in Augenbewegungen und die Untersuchung von Beleuchtungsunterschieden.
Besonders fortschrittlich ist die Methode der Corneal Reflections, bei der die Reflexionen in den Augen einer Person analysiert werden, um Anomalien zu entdecken.

\section*{Auswirkungen und Gegenmaßnahmen}

Die potenziellen Gefahren, die sowohl von Social Engineering als auch von Deepfakes ausgehen, werden in der Ausarbeitung detailliert beschrieben.
Im Falle von Social Engineering wird betont, dass Aufklärung und Schulung der Nutzer eine der wirksamsten Maßnahmen darstellen, um solche Angriffe abzuwehren.
Für den Umgang mit Deepfakes werden technologische Erkennungsmethoden sowie rechtliche Maßnahmen vorgeschlagen, um Missbrauch vorzubeugen.
Gleichzeitig wird die Notwendigkeit unterstrichen, das Bewusstsein der Öffentlichkeit für die Existenz und die Gefahren von Deepfakes zu schärfen, um deren negativen Einfluss auf die Gesellschaft zu minimieren.

\section*{Schlussfolgerung}

 Beide Themen verdeutlichen, wie moderne Technologien und psychologische Manipulationen genutzt werden können, um Schaden anzurichten.
 Um diesen Bedrohungen effektiv zu begegnen, ist es unerlässlich, die Erkennungstechnologien und präventiven Maßnahmen kontinuierlich weiterzuentwickeln.
 Nur durch eine Kombination aus technischer Innovation und umfassender Aufklärung können diese Gefahren eingedämmt werden.
