\section{Praxisbeispiel}\label{sec:praxisbeispiel}
Im Folgenden wird der Workflow mit \gls{dfl} näher betrachtet.
Anschließend werden verschiedenen Konfigurationen und Trainingsdauern verglichen.

\subsection{Laborumgebung}\label{subsec:laborumgebung}
Alle Deepfakes, wenn nicht anders genannt, werden auf folgender Hardware erstellt.\\[0.5cm]
\begin{tabular}{rl}
    CPU:& \texttt{AMD Ryzen 5 2600X}\\
    RAM:& \texttt{16GB DDR4 3000MHz}\\
    GPU:& \texttt{NVIDIA RTX 2070 (8GB GDDR6 VRAM)}\\
    OS:& \texttt{Windows 11}
\end{tabular}\\[0.5cm]

Ziel des Deepfakes ist das Gesicht von \gls{rdj} auf Prof. Volker Knoblauch in diesem \href{https://www.youtube.com/watch?v=rksMPlRSbQU}{Video} zu swappen,
sodass die Hochschule Aalen amerikanische Prominente auf ihrem Youtube-Kanal zeigen kann.

\subsection{Extraction}
Nach der Installation von \gls{dfl} finden sich im entsprechenden Ordner eine Vielzahl von \texttt{.bat}-Dateien, sowie ein \texttt{\_internal}- und \texttt{workspace}-Ordner.
