\section{Bekannte Social Engineering Angriffe}\label{sec:bekannte-social-engineering-angriffe}
\subsection*{Reenacted Video Call in Hong Kong}
Ein besonders eindrucksvolles Beispiel für die Nutzung von Social Engineering in Verbindung mit Deepfake-Technologie
ereignete sich Anfang des Jahres bei einem Unternehmen in Hongkong.
Ein Finanzmitarbeiter wurde von Betrügern dazu gebracht, \$25 Millionen zu überweisen.
Wie die Hongkonger Polizei berichtet, wurde der Mitarbeiter zu einem Videoanruf eingeladen, jede Person in diesem
Meeting war jedoch eine Deepfake. Mittels Echtzeit Face- und Voice-Reenactment wurde das gesamte Meeting gefälscht.

Der Betrug begann mit einer Nachricht, die angeblich vom Finanzchef des Unternehmens in Großbritannien stammte und von einer geheimen Transaktion sprach. Obwohl der Mitarbeiter zunächst misstrauisch war und einen Phishing-Versuch vermutete, ließ er sich durch die anschließende Videokonferenz überzeugen, da die anwesenden Personen wie seine tatsächlichen Kollegen aussahen und klangen.

Dieser Fall zeigt eindrucksvoll, wie fortschrittlich und gefährlich Deepfake-Technologie inzwischen ist. Sie wurde
hier nicht nur verwendet, um ein realistisches Video der Zielperson zu erstellen, sondern auch, um mehrere scheinbar
authentische Teilnehmer an einem Videoanruf zu simulieren. Die Täuschung flog erst auf, als der Mitarbeiter
nachträglich beim Hauptsitz des Unternehmens nachfragte\cite{cnn-deepfake-scam-hong-kong}.

\subsection*{Marriott-Hotel Databreach}
Ein weiteres Beispiel für einen Social Engineering Angriff ereignete sich im Juni 2022 im Marriott-Hotel am Flughafen von Baltimore im US-Bundesstaat Maryland. Kriminelle hatten sich mittels Social Engineering durch einen Mitarbeiter des Hotels Zugang zum Netzwerk verschafft und 20 GB an Daten abgeschöpft, darunter auch Kreditkartendaten von Gästen und interne Geschäftsdaten des Hotels. Marriott hat die Strafverfolgungsbehörden eingeschaltet und wird nach eigenen Abgaben etwa 400 Personen benachrichtigen. Eine Lösegeldforderung der Erpresser lehnte die Hotelkette ab\cite{marriott-data-breach}.

\subsection*{Phishing-Kampanie United States Department of Labor}
Ein weiteres Beispiel für einen Phishing-Angriff ist eine Kampagne aus dem Jahre 2022, die die United States Department
of Labor (DoL) imitiert und Empfänger auffordert, Angebote einzureichen, um Office 365-Anmeldeinformationen zu stehlen.
Die E-Mails passierten gekaperte Server, die gemeinnützigen Organisationen gehören, um E-Mail-Sicherheitsblöcke zu
umgehen. Außerdem wurde die Sender Adresse gespooft um den tatsächlichen Domains des DoL zu entsprechen.

Die Angreifer geben sich als leitender DoL-Mitarbeiter aus, der den Empfänger einlädt, sein Angebot für ein laufendes
Regierungsprojekt einzureichen. Die E-Mails enthalten einen gültigen Briefkopf, professionell gestalteten Inhalt und
einen dreiseitigen PDF-Anhang mit einem scheinbar legitimen Formular.

Das PDF enthält eine „BID“-Schaltfläche, die die Opfer auf eine Phishing-Seite weiterleitet.
Die gefälschte Seite sieht überzeugend aus und enthält identisches HTML und CSS wie die echte. Die Bedrohungsakteure
haben sogar eine Anweisungs-Pop-up-Nachricht hinzugefügt, um die Opfer durch (Phishing-)Prozess zu führen.

Wer für ein Projekt bietet, wird zu einem Formular zur Erfassung von Anmeldeinformationen weitergeleitet, das die
E-Mail-Adresse und das Passwort von Microsoft Office 365 der Opfer abfragt\cite{phishing-dol}.
