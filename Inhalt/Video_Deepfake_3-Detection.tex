\section{Detection Methods für Video-Deepfakes}\label{sec:detection-methods}
Die rasante Verbreitung von Deepfake-Inhalten stellt eine erhebliche Bedrohung für die Privatsphäre, die soziale Sicherheit und die Integrität des Internets dar.
Um diesen Bedrohungen entgegenzuwirken, sind effektive Erkennungsmethoden unerlässlich.
Verschiedene Techniken wurden entwickelt, um die Authentizität von Videos zu prüfen und Deepfakes zu identifizieren.

\subsection*{Temporale Sequenzanalyse}
Eine der gängigsten Methoden zur Erkennung von Deepfakes ist die temporale Sequenzanalyse.
Diese Technik nutzt die Fähigkeit von \gls{lstm} Netzwerken und \glspl{cnn}, um zeitliche Unstimmigkeiten zwischen den Frames eines Videos zu erkennen.
Durch die Analyse der Sequenzen von Frames können \gls{lstm}-Netzwerke zusammen mit \glspl{cnn} Muster identifizieren, die auf Deepfake-Manipulationen hinweisen.
Hierbei extrahieren die \glspl{cnn} eine Vielzahl von Merkmalen aus jedem Frame und übergeben diese an die \gls{lstm}-Netzwerke, die eine temporale Sequenzbeschreibung erzeugen.
Eine SoftMax-Schicht berechnet schließlich die Wahrscheinlichkeit, dass die analysierten Frames Deepfakes sind\cite{Deepfakes-An-Overview}.

\subsection*{Blinzelmuster Erkennung}
Eine weitere Methode basiert auf der Analyse der Augenblinzelmuster in Videos.
Deepfake-Videos weisen oft unnatürliche Blinzelraten auf, da das Blinzeln in den synthetisierten Videos schlecht dargestellt wird.
Hierfür wird das Video in einzelne Frames konvertiert und die Augenbereiche werden extrahiert.
Diese Augenbereich-Sequenzen werden durch \glspl{lrcn} verarbeitet, um die Wahrscheinlichkeit der Augenöffnungs oder -schließzustände vorherzusagen.
Diese Methode ist besonders effektiv, da menschliche Blinzelmuster schwer nachzuahmen und synthetisierbar sind\cite{Deepfakes-An-Overview}.

\subsection*{Physiologisch basierte Erkennungsmethoden}
Physiologisch basierte Erkennungsmethoden nutzen Unterschiede zwischen computergenerierten Gesichtern und realen menschlichen Gesichtern.
Ein Beispiel hierfür ist die Analyse von Blutflussmustern im Gesicht, die in Deepfake-Videos oft fehlen oder unnatürlich dargestellt werden.
Solche physiologischen Signale können mit spezieller Software erfasst und analysiert werden, was eine hohe Genauigkeit bei der Erkennung von subtilen Unterschieden zwischen echten und gefälschten Gesichtern ermöglicht\cite{The-Emergence-of-Deepfake-Technology}.

\subsection*{Digitale Wasserzeichen und Blockchains}
Digitale Wasserzeichen und Blockchain-Technologien bieten ebenfalls effektive Möglichkeiten zur Authentifizierung von Videoinhalten.
Digitale Wasserzeichen können in Videos eingebettet werden und gehen bei Manipulationen verloren.
Die Blockchain-Technologie kann verwendet werden, um digitale Signaturen zu speichern und die Verbreitung von Videos zu verfolgen.
Dies bietet eine hohe Sicherheit und Transparenz bei der Verifizierung der Authentizität von Videoinhalten, erfordert jedoch einen hohen Implementierungsaufwand\cite{The-Emergence-of-Deepfake-Technology}.

\subsection*{Echtzeiterkennung durch Augenspiegelung}
Aktive forensische Methoden sind besonders nützlich für die Echtzeit-Erkennung von Deepfakes, beispielsweise bei Videoanrufen.
Diese Methoden nutzen spezifische Muster oder Reize, um Deepfakes in Echtzeit zu erkennen.
Ein Beispiel ist die Anzeige eines unverwechselbaren Musters auf dem Bildschirm und die Analyse der kornealen Reflexion im Auge des Gesprächspartners.
Solche biometriebasierten Ansätze sind effektiv für die Echtzeit-Erkennung und bieten eine robuste Lösung zur Verhinderung von Deepfake-Angriffen in Videokonferenzen\cite{detection-of-real-time-deepfakes}.\\

Insgesamt bieten die verschiedenen Erkennungsmethoden für Video-Deepfakes eine breite Palette von Ansätzen zur Identifizierung und Authentifizierung von Inhalten.
Die kontinuierliche Weiterentwicklung dieser Technologien ist entscheidend, um den immer leistungsfähigen Deepfake-Techniken entgegenzuwirken.
