% SW
\setcounter{tocdepth}{3}                   % Zeigt Chapter, Section, Subsection in toc
\usepackage{tabularx}
% END SW

\KOMAoptions{DIV      = 10,                % Gibt die Größe des Textbereichs an
             fontsize = 11pt,              % Schriftgröße
             paper    = a4,                % DIN A4-Papier
             parskip  = half}              % Kurzer Abstand statt eingerückter Absatzstart
\setlength{\headheight}{24pt}
\displaywidowpenalty=1000%                 % Leicht erhöhte Strafe für Witwen durch abgesetzten
                                           % Mathematiksatz
\widowpenalty=1000%                        % Leicht erhöhte Strafe für Witwen

\usepackage{iftex}                         % LuaLaTeX wird vorausgesetzt (für fontspec und selnolig)
\RequireLuaTeX%

\usepackage{ragged2e}                      % Für Blocksatz
\usepackage{booktabs}                      % Bessere Tabellenlinien (eigene Befehle, Doku hilft)
\usepackage{csquotes}                      % Bessere Zitate (eigene Befehle, Doku hilft)
\usepackage{relsize}                       % Abgesetzte Zitate mit verkleinerter Schrift
\newenvironment{smallquotes}{\quote\smaller}{\endquote}
\SetBlockEnvironment{smallquotes}
\usepackage{ellipsis}                      % Korrigiert Platz nach \dots
\usepackage{etoolbox}                      % Um das ellipsis-Paket robust zu machen
\robustify\textellipsis%
\usepackage{enumerate}                     % Weitere Möglichkeiten für die enumerate-Umgebung
\usepackage{geometry}                      % Zentriert die Titelseite für die Druckausgabe
\usepackage{graphicx}                      % Einbinden von Bildern
\graphicspath{Bilder/}                      % Bilder können aus dem Verzeichnis img ohne Pfadangabe
                                           % eingebunden werden
\usepackage{icomma}                        % Korrigiert Platz um Kommas in Zahlen
\usepackage{listings}                      % Einbinden von Quelltexten
\usepackage{xcolor}                        % Nutzung von Farben
\lstset{%
    breaklines       = true,               % Automatischer Zeilenumbruch
    backgroundcolor  = \color{gray!8},
    numbers          = left,
    numbersep        = 5pt,
    numberstyle      = \textcolor{black!40},
    showspaces       = false,
    showstringspaces = false,
    tabsize          = 4,
    basicstyle=\small\ttfamily,
    keywordstyle=\bfseries\ttfamily\color{blue},
    stringstyle=\color{orange!50!black}\ttfamily,
    commentstyle=\color{gray}\ttfamily,
    literate=%
        {Ö}{{\"O}}1
        {Ä}{{\"A}}1
        {Ü}{{\"U}}1
        {ß}{{\ss}}1
        {ü}{{\"u}}1
        {ä}{{\"a}}1
        {ö}{{\"o}}1
        {~}{{\textasciitilde}}1
}

\renewcommand*{\lstlistlistingname}{Listings-Verzeichnis}
\usepackage{mathtools}                     % Große Liste von Mathematikbefehlen, enthält amsmath
\usepackage{fontspec}                      % Für den normalen Schriftsatz (standardmäßig: lmodern)
\usepackage[final]{microtype}              % Verbessert die Lesbarkeit und verringert die Anzahl der
                                           % Trennungen
\usepackage{polyglossia}                   % Trennregeln für Deutsch und Englisch
\setmainlanguage[babelshorthands,          % Kürzel für Trennregeln wie im Paket babel
                 spelling = new]{german}
\setotherlanguage{english}
\usepackage[backend = biber,               % Standardmäßig ab TeXLive 2016
            style   = ieee]{biblatex}      % Bibliographie im IEEE-Stil
\addbibresource{Quellen/Literatur.bib}
% Definiert den Bibliographiestil für Online-Quellen nach IEEE StyleGuide von 2009.
% Unter Umständen muss der Stil aktualisiert werden oder kann entfernt werden, wenn er in den
% BibLaTeX-Stil integriert wird.
% https://www.ieee.org/documents/ieeecitationref.pdf
\DefineBibliographyStrings{german}{%
    url = {\mkbibbrackets{Online}\adddot\addspace{}Verfügbar}
}
\renewbibmacro*{title}{%
    \ifboolexpr{
        test {\iffieldundef{title}\fi}
    }\fi
    {}
    {%
        \printtext[title]{%
            \printfield[sentencecase]{title}%
        }%

    }
    \printfield{titleaddon}%
}
\DeclareBibliographyDriver{online}{%
    \usebibmacro{bibindex}%
    \usebibmacro{begentry}%
    \usebibmacro{author/editor+others/translator+others}%
    \adddot\addspace%
    \mkbibparens{\usebibmacro{date}}%
    \adddot\addspace%
    \usebibmacro{title}%
    \addspace%
    \usebibmacro{url}%
}
\usepackage{scrhack}                       % Verbessert die Kompatibilität zwischen KOMAskript und
                                           % einigen älteren Paketen
\usepackage[automark]{scrlayer-scrpage}    % Seitenlayout anpassen
\cohead*{}
\cofoot*{}
\rofoot*{\pagemark}
\usepackage[ngerman]{selnolig}             % Unterdrückt selektiv Ligaturen
\usepackage[ngerman]{varioref}             % Referenziert Labels wo möglich und nötig mit
                                           % relativen Seitenangaben
\usepackage[breaklinks,                    % Trennung von Links erlauben
            colorlinks,                    % Links farbig setzen
            hyperfootnotes = false,        % Sehr fragil und unnötig
            linkcolor      = black,
            urlcolor       = blue]
           {hyperref}                      % Klickbare Hyperlinks
\usepackage[acronym,                       % Abkürzungsverzeichnis erstellen
            nogroupskip,                   % Kein Abstand zwischen Gruppen
            nonumberlist,                  % Keine nummerierte Liste der Einträge
            nopostdot,                     % Kein automatischer Punkt nach Einträgen
            xindy]
            {glossaries}                    % Zum Einbinden eines Abkürzungsverzeichnis und eines
                                           % Glossars
\GlsSetXdyCodePage{duden-utf8}             % Einstellungen für Xindy: Sortierung und UTF8
\setacronymstyle{short-long}               % Erste Abkürzung im Format kurz (lang)
\setglossarystyle{long}                    % Gleiche Breite für Spalten
\setlength\LTleft\parindent%               % Linke Spalte linksbündig statt zentriert
\setlength\LTright\fill%                   % Rechte Spalte füllt die restliche Seite
\loadglsentries{Quellen/glossary.tex}              % Glossareinträge aus dieser Datei lesen
\makeglossaries%                           % Glossareinträge erstellen
\usepackage{cleveref}                      % Referenziert zusätzlich den Typ eines Labels

\AtBeginDocument{% Folgende Befehle am Beginn des Dokuments ausführen
    \hyphenation{Hash-al-go-rith-mus} % Hier globale Regeln zur Silbentrennung einfügen
    \renewcommand*{\acronymname}{Abkürzungsverzeichnis}
}

% Wenn URLs in der Bibliographie unbedingt an jeder Stelle umgebrochen werden können dürfen müssen,
% können die nachfolgenden Zeilen genutzt werden. Achtung: Sehr hässlich. Das \raggedright von unten
% kann in diesem Fall entfernt werden.
% \setcounter{biburlnumpenalty}{1}
% \setcounter{biburlucpenalty}{1}
% \setcounter{biburllcpenalty}{1}

\AtEndDocument{% Folgende Befehle am Ende des Dokuments ausführen
    \cleardoublepage%
    \raggedright%
    % \sloppypar% Kann als Alternative zum \raggedright eingesetzt werden. Blocksatz wie in Word.
    \printbibliography% Bibliographie ausgeben
}

\newcommand*{\Signature}{\Author\\[-0.3cm]\rule{6cm}{1pt}\\Autor}
\newcommand*{\SignatureZwei}{\AuthorZwei\\[-0.3cm]\rule{6cm}{1pt}\\Autor}

\newcommand*{\command}[1]{\texttt{#1}}  % Befehle innerhalb der Zeile in Schreibmaschinenschrift
                                        % setzen
\newcommand*{\definition}[1]{\emph{#1}} % Definitionen in kursiver Schrift setzen
\newcommand*{\filename}[1]{\texttt{#1}} % Dateinamen in Schreibmaschinenschrift setzen
\newcommand*{\programname}[1]{\gls{#1}} % Programme automatisch ins Glossar aufnehmen (nicht
                                        % unbedingt benötigt)
                                        
%\end{document}
