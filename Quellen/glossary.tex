
% im Text mit \gls{Key} oder \glspl{Key} (Plural) verwenden

% Social Engineering
\newacronym{ids}{IDS}{Intrusion Detection System}
\newacronym{dbir}{DBIR}{Data Breach Investigations Report}
\newacronym{osint}{OSINT}{Open Source Intelligence}

\newglossaryentry{threat-intelligence}{
    name={Threat Intelligence},
    description={Threat Intelligence sind Daten, die gesammelt, verarbeitet und analysiert werden, um die Motive,
    Ziele und das Angriffsverhalten eines Bedrohungsakteurs zu verstehen. Durch Threat Intelligence können schnellere,
    fundiertere und datenbasierte Sicherheitsentscheidungen getroffen werden. Zudem ermöglicht es, das Verhalten im
    Kampf gegen Bedrohungsakteure von reaktiv zu proaktiv zu ändern.\cite{crowdstrike-threat-intelligence}}
}
\newglossaryentry{enkeltrick}{
    name={Enkeltrick},
    description={Ein betrügerisches Vorgehen, bei dem sich Trickbetrüger
    über das Telefon, neuerdings auch über Kontaktplattformen und Messengerdienste, meist gegenüber älteren und/oder
    hilflosen Personen, als deren nahe Verwandte ausgeben, um unter Vorspiegelung falscher Tatsachen an deren Bargeld
    oder Wertgegenstände zu gelangen}
}
\newglossaryentry{MaxPooling}{
    name={Max-Pooling},
    description={Max-Pooling ist ein auf Stichproben basierender Diskretisierungsprozess.
    Ziel ist es, eine Eingabedarstellung (Bild, Ausgabematrix der verborgenen Schicht usw.)
    zu verkleinern, um ihre Dimensionalität zu reduzieren und Annahmen über die in den gebinnten
    Unterregionen enthaltenen Merkmale treffen zu können.\cite{ComputerScienceWiki}}
}
\newglossaryentry{VAE}{
    name={Variational Auto-Encoder},
    description={VAEs sind ein Deep Learning-Modell, das verwendet wird, um das generative
    Modellieren von Daten zu erleichtern. VAEs versuchen, ein generisches Modell der Daten zu
    erstellen, indem sie ein generatives Modell aufbauen, das die Daten so gut wie möglich
    beschreibt.\cite{appmeisterei}}
}
\newglossaryentry{Sequenz-zu-Sequenz-Methode}{
    name={Sequenz-zu-Sequenz-Methode},
    description={Bei der Sequenzmodellierung,
    genauer gesagt beim Sequenz-zu-Sequenz-Lernen (Seq2Seq),
    handelt es sich um eine Aufgabe, bei der es darum geht, Modelle zu trainieren,
    um Sequenzen von einer Domäne (z.B. geschriebener Text) in eine andere Domäne
    (z.B. den gleichen Text, der zu Audio synthetisiert wurde) zu konvertieren.\cite{dida}}
}
\newglossaryentry{Magnituden-Spektrogramme}{
    name={Magnituden-Spektrogramme},
    description={Visuelle Darstellungen der Energie von Audiosignalen über die Zeit,
    die für die Sprachsynthese verwendet werden.\cite{ScienceDirectMS}}
}

% DeepFakes
\newacronym{ki}{KI}{Künstliche Intelligenz}
\newacronym{gan}{GAN}{Generative Adversarial Networks}
\newacronym{fsgan}{FSGAN}{Face Swapping GAN}
\newacronym{cnn}{CNN}{Convolutional Neural Network}
\newacronym{lstm}{LSTM}{Long Short-Term Memory}
\newacronym{lrcn}{LRCN}{Long-Term Recurrent Convolutional Network}
\newacronym{sota}{SOTA}{State-Of-The-Art}

% DeepFaceLab
\newacronym{dfl}{DFL}{DeepFaceLab}
\newacronym{cli}{CLI}{Command Line Interface}
\newacronym{rdj}{RDJ}{Robert Downey Jr.}
\newacronym{fps}{FPS}{Frames Per Second}

\newglossaryentry{ai-upscaling}{
    name={AI-Upscaling},
    description={AI-Upscaling ist eine Technik, die künstliche Intelligenz verwendet, um die Auflösung von Bildern oder Videos zu erhöhen.}
}

% DeepFaceLive
\newacronym{jit}{JIT}{Just-in-Time}
\newacronym{dflive}{DFLive}{DeepFaceLive}
\newacronym{obs}{OBS}{Open Broadcaster Software}
\newacronym{rtt}{RTT}{Real Time Transfer}

\newglossaryentry{motiontracking}
{%
    name={Motion Tracking},
    description={Motion Tracking ist eine Technik, die verwendet wird, um die Bewegung von Objekten oder Personen in einem Video oder einer animierten Szene zu verfolgen und zu verfolgen. Dies kann in 2D oder 3D erfolgen.}
}



