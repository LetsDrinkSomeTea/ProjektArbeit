\usepackage{glossaries}% \newacronym{Key}{Short}{Long}

% Glossareinträge
% \newglossaryentry{Key}
% {%
%     name={Name},
%     description={Description}
% }

% im Text mit \gls{Key} oder \glspl{Key} (Plural) verwenden

\newacronym{ids}{IDS}{Intrusion Detection System}
\newacronym{dbir}{DBIR}{Data Breach Investigations Report}

\newglossaryentry{motiontracking}
{%
    name={Motion Tracking},
    description={Motion Tracking ist eine Technik, die verwendet wird, um die Bewegung von Objekten oder Personen in einem Video oder einer animierten Szene zu verfolgen und zu verfolgen. Dies kann in 2D oder 3D erfolgen.}
}

\newglossaryentry{threat-intelligence}{
    name={Threat Intelligence},
    description={Threat Intelligence sind Daten, die gesammelt, verarbeitet und analysiert werden, um die Motive,
Ziele und das Angriffsverhalten eines Bedrohungsakteurs zu verstehen. Durch Threat Intelligence können schnellere,
    fundiertere und datenbasierte Sicherheitsentscheidungen getroffen werden. Zudem ermöglicht es, das Verhalten im
    Kampf gegen Bedrohungsakteure von reaktiv zu proaktiv zu ändern.\cite{crowdstrike-threat-intelligence}}
}

\newglossaryentry{enkeltrick}{
    name={Enkeltrick},
    description={Ein betrügerisches Vorgehen, bei dem sich Trickbetrüger
    über das Telefon, neuerdings auch über Kontaktplattformen und Messengerdienste, meist gegenüber älteren und/oder
    hilflosen Personen, als deren nahe Verwandte ausgeben, um unter Vorspiegelung falscher Tatsachen an deren Bargeld
    oder Wertgegenstände zu gelangen}
}